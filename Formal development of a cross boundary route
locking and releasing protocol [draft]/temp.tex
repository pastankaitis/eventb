\documentclass[11pt,parskip=half]{scrartcl} % Lädt die Einstellungen der Artikel-Dokumentklasse des KOMA-Scripts
\usepackage[utf8]{inputenc} % Ermöglicht die Eingabe von deutschen Sonderzeichen im Quelltext
\usepackage[T1]{fontenc} % Europäische Kodierung der Ausgabeschrift
\usepackage{lmodern} % Lädt die Schriftart Latin Modern, die universell skalierbar ist
\usepackage[english]{babel} % Silbentrennung nach neuer deutscher Rechtschreibung
\usepackage{amsmath} % Mathematik-Paket der American Mathematical Society
\usepackage{graphicx} % Erlaubt das Einfügen von Bildern per \includegraphics{}

\begin{document}
	
	\title{Slotting protocol}
	\subtitle{...}
	\date{\today}
	\maketitle
	
	
	\section{Overview}
	
	The purpose of the protocol is to enable distributed atomic reservation of a collection of resource. For instance, two distinct agents $A_0$ and $A_1$ may request resource collections $\{r_0, r_1, r_2\}$ and $\{r_2, r_3\}$ where $r_0, r_1, r_2, r_3 \in R$. Each agent must get all or nothing - partial request satisfaction is not permitted. Every agent request should be eventually satisfied for as long as certain degenerate situations are avoided. 
	
	The principal mechanism of a solution we offer is \emph{distributed lane}. A distributed lane is a distributed data structure made of, at first approximation, a collection of resource request pools, one per each resource,
	
	An empty distributed lane has the following structure:
	
	$$
	\begin{array}{c|c|c|c|c}
	& r_0 & r_1 & r_2 & r_3 \\
	l_0  \\ \hline
	l_1  \\ \hline
	l_2  \\ \hline
	l_3  \\ \hline
	l_4  \\ \hline
	\end{array}
	$$
	
	where columns represent resources request pools and rows represent lane. Request pools are modelled explicitly with one private pool per resource. Lane are virtual and only present at conceptual level.
	
	\section*{Protocol}
	
	To lock a resource set an agent go through a number of steps defined by the following protocol. We put certain assumption on communication between parties: messages cannot be altered or lost. However, they can be reordered (and hence delayed).
	
	\subsection*{Step 1. Request}
	
	When some agent $A_i$ intends to lock resources $s \subseteq R$ it generates request to request pools associated with resources $\mathbf{r}$. Such requests are sent and received in no particular order and contain only agent name $A_i$. We define such message as $\mathsf{request(A_i)}$ and write $\mathsf{request(A_i) \rightarrow r_k}$ to state it is addressed to resource $r_k \in \mathbf{r}$.
	
	\subsection*{Step 2. Reply}
	
	When a request pool $r_k$ receives request $\mathsf{request(A_i)}$ it increments $\mathsf{pp}(r_k)$ and replies with message $\mathsf{reply(\mathsf{pp}(r_k))} \rightarrow A_i$. Value $\mathsf{pp}(r_k)$ is the \emph{promise pointer} of request pool $r_k$. The pointer is the largest \emph{promised} index in request pool. The initial value $\mathsf{pp}(r_k)$ is unique for every request pool and is assigned statically.
	
	Notice that addressed message $\mathsf{reply(\mathsf{pp}(r_k))} \rightarrow A_i$ does not carry information about the sender.
	
	\subsection*{Step 3. Confirm}
	
	After sending $\mathsf{request(A_i)}$, agent $A_i$ awaits for all replies to arrive; the reply messages carry values $\mathsf{pp}(r_k)$. Once all messages are received, the agent computes $n=max_{r \in \mathbf{r}} \mathsf{pp}(r)$. This value is used to generate message $\mathsf{reserve(A_i, n)} \rightarrow r, r \in \mathbf{r}$, carrying maximum value $n$ and addressed to all the pools.
	
	\subsection*{Step 3-4. (Paulius)}
	
	In this version one can combine Steps 3 and 4. Since the approach of this protocol version is to iteratively try to form a lane. A basic solution could be as follows:
	
	\begin{enumerate}
		\item After receiving all replies one can check  $\mathsf \forall e \cdot e \in reply[\{A_i\}]^{-1} \Rightarrow e = n$
		\begin{enumerate}
			\item If true proceed to Step 5a.
			\item Else prooced to Step 5b.
		\end{enumerate}
	\end{enumerate}
	
	\textbf{Remark}: In order to allow $\mathsf \forall e \cdot e \in reply[\{A_i\}]^{-1} \Rightarrow e \leq n$ one must make an assumption on request arriving frequency. 
	
	
	
	\subsection*{Step 4. Accept/Reply (Alexei)}
	
	When a request pool $r_k$ receives message $\mathsf{reserve(A_i, n)}$ it proceeds in one of the two ways:
	\begin{enumerate}
		\item if $n$ is equal or greater than the current value of \emph{promise pointer} $\mathsf{pp}(r_k)$ then the pool replies with message $\mathsf{accept}(n) \rightarrow A_i$;
		\item if $n$ is less than the current value of \emph{promise pointer} $\mathsf{pp}(r_k)$ then the pool replies with message $\mathsf{reject(\mathsf{pp}(r_k))} \rightarrow A_i$.
	\end{enumerate}
	
	
	
	\subsection*{Step 5.a. Write (request)}
	
	Agent $A_i$ again awaits for all replies to arrive. In a general case, these are made of  $\mathsf{accept}(n)$ (all $n$ values are same by the nature of the protocol) and  $\mathsf{reject}$ messages. This step applies only to the situation when there are no rejections.
	
	Since all the pools replied with the acceptance, it is now possible to write into all the pools and thus form a new lane. The agent replies with message $\mathsf{write}(A_i, n) \rightarrow r_k$ to each pool $r_k$.
	
	\subsection*{Step 5.b. Restart}
	
	If any reply message from pools is a rejection message then the agent must re-negotiate index for a new lane. The current protocol round is abandoned and the agent starts with Step 1. 
	
	\subsection*{Step 6. Write}
	
	On reception of $\mathsf{write}(A_i, n)$, request pool writes $A_i$ at index $n$.
	
	
	\subsection*{Consumption}
	
	The protocol steps above describe how to form lanes from distributed pools. The pragmatics of such lanes is atomic reservation of resource sets. The lanes semantics guarantees, as we shall prove in the model, the following two properties:
	
	\begin{enumerate}
		\item every reserved value $A_i$ forms a valid lane: it has same index in each request pool;
		\item the smallest index of all request pools (i.e., the lane with a smallest index) define a valid set of requested resources that can be locked;
		\item there are no deadlocks.
	\end{enumerate}
	
	
	
\end{document} 
