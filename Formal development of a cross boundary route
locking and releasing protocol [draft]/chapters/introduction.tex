\section{Introduction}
The railway interlocking system is essential for ensuring safe operation of railway networks. Nowadays a computer based system is used to establish safe routes trains can pass through. A railway intelocking system is capable of controlling only a portion of railway network therefore multiple interlocking systems have to be used to guarantee safety of the whole network. Often a train is required to transition from one interlocking area to another and cross boundary route locking and releasing protocol must ensure a safe transition. Because of distributed system complexity rigourous and systematic verification methods have to be used to guarantee correctness of such protocols.

Formal methods - a mathematical model driven methods provide a systematic approach for developing complex systems. They offer an approach to specify systems precisely via a mathematically defined syntax and semantics as well as formally validate them by using semi-automatic or automatic techniques. In spite of formal methods success in the railway domain considerably little has been done in considering distributed nature of the railway interlocking. In this paper we present a formal development of the cross boundary route locking and releasing protocol in the Event-B \cite{EventBBook} specification language. 

The main protocol objective is to guarantee a safe rolling stock transition from one interlocking area to the following one. In addition to safety properties one should also guarantee liveness and deadlock-freedom of the system as well. The literature review (see below) showed that based on the engineering concept two main distributed railway models exist. The first concept of the interlocking system (described in the first paragraph) controls a number of railway agents and is more closer to real-world interlocking systems. Whereas the second is more fined grained interlocking system where each component has an autonomy to make a decision. %\textbf{Our approach is closer to the first one..}


In the following section we describe Event-B specification language which was used to develop formal model and prove the properties of interest. In Section \ref{section3} we informally describe the cross boundary route locking and releasing protocol including key requirements and environment assumptions. In Sections \ref{fmodel} - \ref{proof} we present the formal model of the protocol and verification challenges. In the last section we discuss the modelling and outline concluding remarks. \\


\noindent \textbf{Related Work.} To authors knowledge the earliest attempt to formally analyse distributed railway solid-state interlocking systems was completed by Morley \cite{morley1996safety}. In this interesting work author developed a formal model of a protocol for a cross boundary route locking and releasing mechanism. By analysing temporal properties of the model he discovered that in certain scenarios safety properties can be violated. Few years later a paper by Cimatti et al. \cite{cimatti1999formal} presented an industrially driven formal methods study where authors formally modelled a communication protocol for safety-critical distributed systems including distributed railway interlocking systems. Their method used Statecharts diagrams to specify high level protocol properties and the \scriptsize{OBJECT}\normalsize{GEODE} model checker for the protocol validation. 

In other work a different concept of distributed railway control system was introduced by Haxthausen and Peleska \cite{haxthausen2000formal}. Their presented engineering concept of the control system relied on a radio based communication and switch boxes - systems which can only control a single railway switch. Authors formally modelled the system with the RAISE \cite{george1995raise} specification language  which allowed to develop a formal model incrementally using a refinement process and prove refinement and safety properties with available justification tools. The timing properties of the design were considered in the extended work \cite{madsen2005modelling}. Similar ideas for distributed railway interlocking system were also presented in \cite{banci2004role, hei2008toward} where authors used Statecharts and Petri Nets to model and verify decentralised railway interlocking. 
