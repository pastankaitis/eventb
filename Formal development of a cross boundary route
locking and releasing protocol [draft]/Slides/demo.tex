\documentclass[10pt]{beamer}
\usepackage{tikz}
\usetikzlibrary{arrows.meta}
\usetikzlibrary{fit}

\usetikzlibrary{arrows,positioning, calc,lindenmayersystems,decorations.pathmorphing,intersections}
\tikzstyle{resource}= [draw,minimum size=16pt,inner sep=0pt]
\tikzstyle{process} = [draw,minimum size=16pt,inner sep=0pt,circle]
\tikzstyle{allocated} = [->,thick,arrows={-latex}]
\tikzstyle{requested} = [<-,thick,arrows={latex-}, dashed]

\usepackage{tikz}



\usetikzlibrary{positioning}
\newcommand{\MonetaryLevel}{Monetary level}
\newcommand{\RealLevel}{Real level}
\newcommand{\Firms}{Firms}
\newcommand{\Households}{Households}
\newcommand{\Banks}{Banks}
\newcommand{\Commodities}{Commodities}
\newcommand{\LaborPower}{Labor power}
\newcommand{\Wages}{Wages}
\newcommand{\Consumption}{Consumption}
\newcommand{\Credits}{Credits}
\newcommand{\Withdrawals}{Withdrawals}
\newcommand{\Deposits}{Deposits}
\newcommand{\Repayments}{Repayments}

\newcommand{\yslant}{0.5}
\newcommand{\xslant}{-0.6}



\newcommand{\VALID}		{\mathsf{valid}}
\newcommand{\INVALID}	{\mathsf{invalid}}
\newcommand{\UNKNOWN}	{\mathsf{unknown}}
\newcommand{\FAILED}	{\mathsf{failed}}
\newcommand{\prover}	{\mathrm{prover}}
\newcommand{\INTER}[1] 	{\llbracket #1 \rrbracket}

\providecommand{\keywords}[1]{\textbf{{Keyword:}} #1}

\usepackage{bsymb}
\input{railheader}

\newtheorem*{remark}{PhD Objective.}



\usetheme[progressbar=frametitle]{metropolis}
\usepackage{appendixnumberbeamer}

\usepackage{booktabs}
\usepackage[scale=2]{ccicons}

\usepackage{pgfplots}
\usepgfplotslibrary{dateplot}

\newcommand\blfootnote[1]{%
	\begingroup
	\renewcommand\thefootnote{}\footnote{#1}%
	\addtocounter{footnote}{-1}%
	\endgroup
}

\usepackage{xspace}
\newcommand{\themename}{\textbf{\textsc{metropolis}}\xspace}


\definecolor{darkblue}{rgb}{0,0,0.3}
\definecolor{lightblue}{rgb}{0.3,0.3,1}
\definecolor{darkred}{rgb}{0.7,0,0}
\definecolor{lightred}{rgb}{1,0.3,0.3}
\definecolor{orange}{rgb}{0.6,0.1,0.1}
\newcommand{\VAR}[1] {\textcolor{darkblue}{\mathit{#1}}}

\newcommand{\PAR}[1] {\textcolor{lightblue}{\mathit{#1}}}
\newcommand{\CST}[1] {\textcolor{lightred}{\mathsf{#1}}}
\input{convheader}

\usepackage{listings}
\usepackage{color}

\definecolor{codegreen}{rgb}{0,0.6,0}
\definecolor{codegray}{rgb}{0.5,0.5,0.5}
\definecolor{codepurple}{rgb}{0.58,0,0.82}
\definecolor{backcolour}{rgb}{0.95,0.95,0.92}

\lstdefinestyle{mystyle}{
	backgroundcolor=\color{backcolour},   
	commentstyle=\color{codegreen},
	keywordstyle=\color{magenta},
	numberstyle=\tiny\color{codegray},
	stringstyle=\color{codepurple},
	basicstyle=\scriptsize,
	breakatwhitespace=false,         
	breaklines=true,                 
	captionpos=b,                    
	keepspaces=true,                 
	numbers=left,                    
	numbersep=5pt,                  
	showspaces=false,                
	showstringspaces=false,
	showtabs=false,                  
	tabsize=2
}

\lstset{style=mystyle}

\title{Theories, Techniques and Tools for Engineering
	Heterogeneous Railway Networks}


%\date{}
\author{Paulius Stankaitis and Alexei Iliasov}
\institute{Centre for Software Reliability, Newcastle University, UK \vspace{0.2cm}\\ \begin{center} \small
		RSSRail Conference '17\vspace{0.1cm}\\ November 16th, Pistoia
	\end{center}}
\titlegraphic{\hfill\includegraphics[height=0.8cm]{Newcastle_University_Logo.pdf}}
\date{}
\begin{document}




 


\begin{frame}{Dining Philosophers}
	\begin{center}
		\begin{tikzpicture}[scale=.75,every node/.style={minimum size=1cm},on grid]
		
		% Real level
		
		

		\node (r1)[resource] at (6, 3.7) {$R_{1}$};
		\node (r4)[resource] at (2,3.7) {$R_{4}$};
		\node (r2)[resource] at (6,.3) {$R_{2}$};
		\node (r3)[resource] at (2,.3) {$R_{3}$};
		
		\node (p1)[process]  at (4,5) {$P_1$};
		\node (p2)[process]  at (7,2) {$P_2$};
		\node (p4)[process]  at (1,2) {$P_4$};
		\node (p3)[process]  at (4,-1) {$P_3$};
		% Agents:
		
		\draw[requested] (r1) -- (p1);
		\draw[requested] (r4) -- (p1);
		
		\draw[requested] (r1) -- (p2);
		\draw[requested] (r2) -- (p2);

		\draw[requested] (r2) -- (p3);
		\draw[requested] (r3) -- (p3);	
		
		\draw[requested] (r4) -- (p4);
		\draw[requested] (r3) -- (p4);		
	
		\end{tikzpicture}
		
	\end{center}
\end{frame}

\begin{frame}{Dining Philosophers - Assumptions}
	\begin{center}
		

	\begin{tabular}{ c|c| c |c | c |} 
		
		&	r$_1$ & r$_2$ & r$_3$ & r$_4$ \\  
		1 &	 &  & & \\ 
		2 &	 &  &  &\\ 
		3 &	 &  &  &\\ 
		\vdots &	\vdots & \vdots  & \vdots  & \vdots\\ 
		n & & & &
		
	\end{tabular}
		
	\end{center}
	
\end{frame}

\begin{frame}{Dining Philosophers - Protocol}
	\begin{center}
		

	\begin{tikzpicture}[scale=.7,every node/.style={minimum size=1cm},on grid]

			\node (p2)[process]  at (7,2) {$P_2$};
			\node (r1)[resource] at (6,5) {$R_{1}$};		
			\node (r2)[resource] at (6,-1) {$R_{2}$};

			
			\draw[requested] (r1) -- (p2) node [pos=0.5,right,font=\footnotesize] {ask};

			\draw[requested] (r2) -- (p2) node [pos=0.5,right,font=\footnotesize] {ask};	

	\end{tikzpicture}
	\end{center}	
	
	
\end{frame}

\begin{frame}{Dining Philosophers - Protocol}
	\begin{center}
		
		
		\begin{tikzpicture}[scale=.7,every node/.style={minimum size=1cm},on grid]
		
		\node (p2)[process]  at (7,2) {$P_2$};
		\node (r1)[resource] at (6,5) {$R_{1}$};		
		\node (r2)[resource] at (6,-1) {$R_{2}$};
		
		
		\draw[requested] (p2) -- (r1) node [pos=0.5,right,font=\footnotesize] {reply(pp$_1$)};
		
		\draw[requested] (p2) -- (r2) node [pos=0.5,right,font=\footnotesize] {reply(pp$_2$)};	
		
		\end{tikzpicture}
	\end{center}	
	
	
\end{frame}

\begin{frame}{Dining Philosophers - Protocol}
	\begin{center}
		
		
		\begin{tikzpicture}[scale=.7,every node/.style={minimum size=1cm},on grid]
		
		\node[label={[xshift=1.2cm, yshift=-.75cm]confirm?}] (p2)[process]  at (7,2) {$P_2$};
		\node (r1)[resource] at (6,5) {$R_{1}$};		
		\node (r2)[resource] at (6,-1) {$R_{2}$};
		
	
		
				\draw[requested] (r1) -- (p2);
				\draw[requested] (r2) -- (p2);	
		
		\end{tikzpicture}
	\end{center}	
	
\end{frame}

\begin{frame}{Dining Philosophers - Protocol}
	\begin{center}
		
		
		\begin{tikzpicture}[scale=.7,every node/.style={minimum size=1cm},on grid]
		
		\node[label={[xshift=3.0cm, yshift=-.75cm]\small compute max(reply(pp$_1$), reply(pp$_2$))}] (p2)[process]  at (7,2) {$P_2$};
		\node (r1)[resource] at (6,5) {$R_{1}$};		
		\node (r2)[resource] at (6,-1) {$R_{2}$};
		
		
		
		\draw[requested] (r1) -- (p2);
		\draw[requested] (r2) -- (p2);	
		
		\end{tikzpicture}
	\end{center}	
	
\end{frame}

\begin{frame}{Dining Philosophers - Protocol}
	\begin{center}
		
		
		\begin{tikzpicture}[scale=.7,every node/.style={minimum size=1cm},on grid]
		
		\node (p2)[process]  at (7,2) {$P_2$};
		\node (r1)[resource] at (6,5) {$R_{1}$};		
		\node (r2)[resource] at (6,-1) {$R_{2}$};
		
		
		\draw[requested] (r1) -- (p2) node [pos=0.5,right,font=\footnotesize] {lock(p$_2$, max)};
		
		\draw[requested] (r2) -- (p2) node [pos=0.5,right,font=\footnotesize] {lock(p$_2$, max)};	
		
		\end{tikzpicture}
	\end{center}	
	
\end{frame}

\begin{frame}{Dining Philosophers - Protocol}
	\begin{center}
		
		
		\begin{tikzpicture}[scale=.7,every node/.style={minimum size=1cm},on grid]
		
		\node (p2)[process]  at (7,2) {$P_2$};
		\node (r1)[resource] at (6,5) {$R_{1}$};		
		\node (r2)[resource] at (6,-1) {$R_{2}$};
		
		
		\draw[requested] (r1) -- (p2) node [pos=0.5,right,font=\footnotesize] {capture/release};
		
		\draw[requested] (r2) -- (p2) node [pos=0.5,right,font=\footnotesize] {capture/release};	
		
		\end{tikzpicture}
	\end{center}	
\end{frame}

\begin{frame}{Dining Philosophers - Problems}
	\textbf{Problem$_1$}: qpp(r$_n$) is incremented (at reply) but if not used (at confirm) points to the wrong location. Incrementing qpp(r$_n$) at confirm would result in multiple request receiving the same qpp(r$_n$). \vspace{.5cm} \\ \textbf{Fix}: update qpp(r$_n$) again at locking;
\end{frame}

\begin{frame}{Dining Philosophers - Problems}
	\textbf{Problem$_2$}: Suppose, p$_1$ requested \{r$_1$, r$_2$\}. \\ r$_1$ replies p$_1$ with n$_1$ and before r$_2$ replies with n$_2$ (n$_1$ $<$ n$_2$); \\
	 Suppose pp(r$_1$) has been incremented few times (because someone else also asked for the same resource). Thus when eventually he gets both replies and computes max(n$_1$, n$_2$) = n$_2$ for p$_1$ cannot write to r$_1$ at r$_1($n$_2$).   \\ 
	
	\textbf{Fix}: 
\end{frame}

\begin{frame}{Dining Philosophers - Problems}
	\textbf{Problem$_3$}: Confirming that all requests and reply messages have been received, and a maximum number can be returned. \vspace{.5cm} \\ 
	
	\textbf{Fix}: 
\end{frame}



	







\end{document}
