\documentclass[10pt]{beamer}
\usepackage{tikz}
\usetikzlibrary{arrows.meta}
\usetikzlibrary{fit}

\usetikzlibrary{arrows,positioning, calc,lindenmayersystems,decorations.pathmorphing,intersections}
\tikzstyle{resource}= [draw,minimum size=16pt,inner sep=0pt]
\tikzstyle{process} = [draw,minimum size=16pt,inner sep=0pt,circle]
\tikzstyle{allocated} = [->,thick,arrows={-latex}]
\tikzstyle{requested} = [<-,thick,arrows={latex-}, dashed]

\usepackage{tikz}



\usetikzlibrary{positioning}
\newcommand{\MonetaryLevel}{Monetary level}
\newcommand{\RealLevel}{Real level}
\newcommand{\Firms}{Firms}
\newcommand{\Households}{Households}
\newcommand{\Banks}{Banks}
\newcommand{\Commodities}{Commodities}
\newcommand{\LaborPower}{Labor power}
\newcommand{\Wages}{Wages}
\newcommand{\Consumption}{Consumption}
\newcommand{\Credits}{Credits}
\newcommand{\Withdrawals}{Withdrawals}
\newcommand{\Deposits}{Deposits}
\newcommand{\Repayments}{Repayments}

\newcommand{\yslant}{0.5}
\newcommand{\xslant}{-0.6}



\newcommand{\VALID}		{\mathsf{valid}}
\newcommand{\INVALID}	{\mathsf{invalid}}
\newcommand{\UNKNOWN}	{\mathsf{unknown}}
\newcommand{\FAILED}	{\mathsf{failed}}
\newcommand{\prover}	{\mathrm{prover}}
\newcommand{\INTER}[1] 	{\llbracket #1 \rrbracket}

\providecommand{\keywords}[1]{\textbf{{Keyword:}} #1}

\usepackage{bsymb}
\input{railheader}

\newtheorem*{remark}{PhD Objective.}



\usetheme[progressbar=frametitle]{metropolis}
\usepackage{appendixnumberbeamer}

\usepackage{booktabs}
\usepackage[scale=2]{ccicons}

\usepackage{pgfplots}
\usepgfplotslibrary{dateplot}

\newcommand\blfootnote[1]{%
	\begingroup
	\renewcommand\thefootnote{}\footnote{#1}%
	\addtocounter{footnote}{-1}%
	\endgroup
}

\usepackage{xspace}
\newcommand{\themename}{\textbf{\textsc{metropolis}}\xspace}


\definecolor{darkblue}{rgb}{0,0,0.3}
\definecolor{lightblue}{rgb}{0.3,0.3,1}
\definecolor{darkred}{rgb}{0.7,0,0}
\definecolor{lightred}{rgb}{1,0.3,0.3}
\definecolor{orange}{rgb}{0.6,0.1,0.1}
\newcommand{\VAR}[1] {\textcolor{darkblue}{\mathit{#1}}}

\newcommand{\PAR}[1] {\textcolor{lightblue}{\mathit{#1}}}
\newcommand{\CST}[1] {\textcolor{lightred}{\mathsf{#1}}}
\input{convheader}

\usepackage{listings}
\usepackage{color}

\definecolor{codegreen}{rgb}{0,0.6,0}
\definecolor{codegray}{rgb}{0.5,0.5,0.5}
\definecolor{codepurple}{rgb}{0.58,0,0.82}
\definecolor{backcolour}{rgb}{0.95,0.95,0.92}

\lstdefinestyle{mystyle}{
	backgroundcolor=\color{backcolour},   
	commentstyle=\color{codegreen},
	keywordstyle=\color{magenta},
	numberstyle=\tiny\color{codegray},
	stringstyle=\color{codepurple},
	basicstyle=\scriptsize,
	breakatwhitespace=false,         
	breaklines=true,                 
	captionpos=b,                    
	keepspaces=true,                 
	numbers=left,                    
	numbersep=5pt,                  
	showspaces=false,                
	showstringspaces=false,
	showtabs=false,                  
	tabsize=2
}

\lstset{style=mystyle}

\title{Theories, Techniques and Tools for Engineering
	Heterogeneous Railway Networks}


%\date{}
\author{Paulius Stankaitis and Alexei Iliasov}
\institute{Centre for Software Reliability, Newcastle University, UK \vspace{0.2cm}\\ \begin{center} \small
		RSSRail Conference '17\vspace{0.1cm}\\ November 16th, Pistoia
	\end{center}}
\titlegraphic{\hfill\includegraphics[height=0.8cm]{Newcastle_University_Logo.pdf}}
\date{}
\begin{document}

\section{Scenario 1}

\begin{frame}{Dining Philosophers}
	\begin{center}
		
		\begin{tikzpicture}[scale=.5,every node/.style={minimum size=1cm},on grid]
		
		\node (p0)[process]  at (4,2) {$p_0$};
		\node (p1)[process]  at (8,2) {$p_1$};
		\node (r0)[resource] at (2,-1) {$r_{0}$};		
		\node (r1)[resource] at (6,-1) {$r_{1}$};
		\node (r2)[resource] at (10,-1) {$r_{2}$};
		
		\draw[requested] (r0) -- (p0) node [pos=0.1,right,font=\footnotesize] {q$_0$};
		\draw[requested] (r1) -- (p0) node [pos=0.1,left,font=\footnotesize] {q$_0$};
		
		\draw[requested] (r1) -- (p1) node [pos=0.1,right,font=\footnotesize] {q$_1$};
		\draw[requested] (r2) -- (p1) node [pos=0.1,left,font=\footnotesize] {q$_1$};
		
		
		\end{tikzpicture}
	\end{center}	
	
	\begin{center}
		
		
		\begin{tabular}{ c c| c c |c c} 
			
			&	r$_0$ &  & r$_1$ & & r$_2$   \\  
			\hline 
			0 &	  q$_0$*    & 1  &        & 2 &	\\ 
			1 &		  	   & 2  & 	    & 3	&			 \\ 
			2 &	      	   & 3  &        & 4 &                  \\ 
			\vdots &	\vdots & \vdots  & \vdots  & \vdots\\ 
			n & & n & &n
			
		\end{tabular}
		
	\end{center}
\textbf{Comment}: Suppose r$_0$ replies to p$_0$ first with 0 and increments pp(r$_0$) to pp(r$_0$) = 1 (* means promised, not locked yet).
\end{frame}

\begin{frame}{Dining Philosophers}
	\begin{center}
		
		\begin{tikzpicture}[scale=.5,every node/.style={minimum size=1cm},on grid]
		
		\node (p0)[process]  at (4,2) {$p_0$};
		\node (p1)[process]  at (8,2) {$p_1$};
		\node (r0)[resource] at (2,-1) {$r_{0}$};		
		\node (r1)[resource] at (6,-1) {$r_{1}$};
		\node (r2)[resource] at (10,-1) {$r_{2}$};
		
		\draw[requested] (r0) -- (p0) node [pos=0.1,right,font=\footnotesize] {q$_0$};
		\draw[requested] (r1) -- (p0) node [pos=0.1,left,font=\footnotesize] {q$_0$};
		
		\draw[requested] (r1) -- (p1) node [pos=0.1,right,font=\footnotesize] {q$_1$};
		\draw[requested] (r2) -- (p1) node [pos=0.1,left,font=\footnotesize] {q$_1$};
		
		
		\end{tikzpicture}
	\end{center}	
	
	\begin{center}
		
		
		\begin{tabular}{ c c| c c |c c} 
			
			&	r$_0$ &  & r$_1$ & & r$_2$   \\  
			\hline 
			0 &	  q$_0$*    & 1  &  q$_1$*       & 2 &	\\ 
			1 &		  	   & 2  & 	    & 3	&			 \\ 
			2 &	      	   & 3  &        & 4 &                  \\ 
			\vdots &	\vdots & \vdots  & \vdots  & \vdots\\ 
			n & & n & &n
			
		\end{tabular}
		
	\end{center}
	\textbf{Comment}: Suppose r$_1$ replies to p$_1$ with 1 and increments pp(r$_1$) to pp(r$_1$) = 2.
\end{frame}

\begin{frame}{Dining Philosophers}
	\begin{center}
		
		\begin{tikzpicture}[scale=.5,every node/.style={minimum size=1cm},on grid]
		
		\node (p0)[process]  at (4,2) {$p_0$};
		\node (p1)[process]  at (8,2) {$p_1$};
		\node (r0)[resource] at (2,-1) {$r_{0}$};		
		\node (r1)[resource] at (6,-1) {$r_{1}$};
		\node (r2)[resource] at (10,-1) {$r_{2}$};
		
		\draw[requested] (r0) -- (p0) node [pos=0.1,right,font=\footnotesize] {q$_0$};
		\draw[requested] (r1) -- (p0) node [pos=0.1,left,font=\footnotesize] {q$_0$};
		
		\draw[requested] (r1) -- (p1) node [pos=0.1,right,font=\footnotesize] {q$_1$};
		\draw[requested] (r2) -- (p1) node [pos=0.1,left,font=\footnotesize] {q$_1$};
		
		
		\end{tikzpicture}
	\end{center}	
	
	\begin{center}
		
		
		\begin{tabular}{ c c| c c |c c} 
			
			&	r$_0$ &  & r$_1$ & & r$_2$   \\  
			\hline 
			0 &	  q$_0$*    & 1  &  q$_1$*       & 2 & q$_1$*	\\ 
			1 &		  	   & 2  & 	    & 3	& 			 \\ 
			2 &	      	   & 3  &        & 4 &                  \\ 
			\vdots &	\vdots & \vdots  & \vdots  & \vdots\\ 
			n & & n & &n
			
		\end{tabular}
		
	\end{center}
	\textbf{Comment}: Suppose r$_2$ replies to p$_1$ with 2 and increments pp(r$_2$) to pp(r$_1$) = 3.
\end{frame}

\begin{frame}{Dining Philosophers}
	\begin{center}
		
		\begin{tikzpicture}[scale=.5,every node/.style={minimum size=1cm},on grid]
		
		\node (p0)[process]  at (4,2) {$p_0$};
		\node (p1)[process]  at (8,2) {$p_1$};
		\node (r0)[resource] at (2,-1) {$r_{0}$};		
		\node (r1)[resource] at (6,-1) {$r_{1}$};
		\node (r2)[resource] at (10,-1) {$r_{2}$};
		
		\draw[requested] (r0) -- (p0) node [pos=0.1,right,font=\footnotesize] {q$_0$};
		\draw[requested] (r1) -- (p0) node [pos=0.1,left,font=\footnotesize] {q$_0$};
		
		\draw[requested] (r1) -- (p1) node [pos=0.1,right,font=\footnotesize] {q$_1$};
		\draw[requested] (r2) -- (p1) node [pos=0.1,left,font=\footnotesize] {q$_1$};
		
		
		\end{tikzpicture}
	\end{center}	
	
	\begin{center}
		
		
		\begin{tabular}{ c c| c c |c c} 
			
			&	r$_0$ &  & r$_1$ & & r$_2$   \\  
			\hline 
			0 &	  q$_0$*    & 1  &  q$_1$*       & 2 & q$_1$*	\\ 
			1 &		  	   & 2  & 	q$_0$*    & 3	& 			 \\ 
			2 &	      	   & 3  &        & 4 &                  \\ 
			\vdots &	\vdots & \vdots  & \vdots  & \vdots\\ 
			n & & n & &n
			
		\end{tabular}
		
	\end{center}
	\textbf{Comment}: Suppose r$_1$ replies to p$_0$ with 2 and increments pp(r$_2$) to pp(r$_1$) = 3.
\end{frame}

\begin{frame}{Dining Philosophers}
	\begin{center}
		
		\begin{tikzpicture}[scale=.5,every node/.style={minimum size=1cm},on grid]
		
		\node (p0)[process]  at (4,2) {$p_0$};
		\node (p1)[process]  at (8,2) {$p_1$};
		\node (r0)[resource] at (2,-1) {$r_{0}$};		
		\node (r1)[resource] at (6,-1) {$r_{1}$};
		\node (r2)[resource] at (10,-1) {$r_{2}$};
		
		\draw[requested] (r0) -- (p0) node [pos=0.1,right,font=\footnotesize] {q$_0$};
		\draw[requested] (r1) -- (p0) node [pos=0.1,left,font=\footnotesize] {q$_0$};
		
		\draw[requested] (r1) -- (p1) node [pos=0.1,right,font=\footnotesize] {q$_1$};
		\draw[requested] (r2) -- (p1) node [pos=0.1,left,font=\footnotesize] {q$_1$};
		
		
		\end{tikzpicture}
	\end{center}	
	
	\begin{center}
		
		
		\begin{tabular}{ c c| c c |c c} 
			
			&	r$_0$ &  & r$_1$ & & r$_2$   \\  
			\hline 
			0 &	  q$_0$*    & 1  &  q$_1$*       & 2 & q$_1$*	\\ 
			1 &		  	   & 2  & 	q$_0$*    & 3	& 			 \\ 
			2 &	      	   & 3  &        & 4 &                  \\ 
			\vdots &	\vdots & \vdots  & \vdots  & \vdots\\ 
			n & & n & &n
			
		\end{tabular}
		
	\end{center}
	\textbf{Comment}: Both p$_0$ and p$_1$ can confirm they received all replies they needed, but max(p$_0$) = max(p$_1$) = 2 and so they both would like to lock r$_1$(2).
\end{frame}

\section{Scenario 2}
 \begin{frame}{Dining Philosophers}
	\begin{center}
		
		\begin{tikzpicture}[scale=.5,every node/.style={minimum size=1cm},on grid]
		
		\node (p0)[process]  at (0,2) {$p_0$};
		\node (p1)[process]  at (4,2) {$p_1$};
		\node (p2)[process]  at (8,2) {$p_2$};
		\node (r0)[resource] at (0,-1) {$r_{0}$};		
		\node (r1)[resource] at (5,-1) {$r_{1}$};
		\node (r2)[resource] at (10,-1) {$r_{2}$};
		
		\draw[requested] (r0) -- (p0);
		\draw[requested] (r1) -- (p0);
		
		\draw[requested] (r2) -- (p1);
		\draw[requested] (r0) -- (p1);
		
		\draw[requested] (r2) -- (p2);
		\draw[requested] (r0) -- (p2);
		
		
		\end{tikzpicture}
	\end{center}	
	
	\begin{center}
		
		
		\begin{tabular}{ c c| c c |c c} 
			
			&	r$_0$ &  & r$_1$ & & r$_2$   \\  
			\hline 
			0 &	 q$_0$*     & 1  &        & 2 &	\\ 
			1 &		  	   & 2  & 	    & 3	&			 \\ 
			2 &	      	   & 3  &        & 4 &                  \\ 
			\vdots &	\vdots & \vdots  & \vdots  & \vdots\\ 
			n & & n & &n
			
		\end{tabular}
		
	\end{center}
	\textbf{Comment}: Suppose r$_0$ replies to p$_0$ with 0 and increments pp(r$_0$) to pp(r$_0$) = 1.
\end{frame}

\begin{frame}{Dining Philosophers}
	\begin{center}
		
		\begin{tikzpicture}[scale=.5,every node/.style={minimum size=1cm},on grid]
		
		\node (p0)[process]  at (0,2) {$p_0$};
		\node (p1)[process]  at (4,2) {$p_1$};
		\node (p2)[process]  at (8,2) {$p_2$};
		\node (r0)[resource] at (0,-1) {$r_{0}$};		
		\node (r1)[resource] at (5,-1) {$r_{1}$};
		\node (r2)[resource] at (10,-1) {$r_{2}$};
		
		\draw[requested] (r0) -- (p0);
		\draw[requested] (r1) -- (p0);
		
		\draw[requested] (r2) -- (p1);
		\draw[requested] (r0) -- (p1);
		
		\draw[requested] (r2) -- (p2);
		\draw[requested] (r0) -- (p2);
		
		
		\end{tikzpicture}
	\end{center}	
	
	\begin{center}
		
		
		\begin{tabular}{ c c| c c |c c} 
			
			&	r$_0$ &  & r$_1$ & & r$_2$   \\  
			\hline 
			0 &	 q$_0$*     & 1  &        & 2 &	\\ 
			1 &	q$_1$*	  	   & 2  & 	    & 3	&			 \\ 
			2 &	 p$_2$*     	   & 3  &        & 4 &                  \\ 
			3 &	     	   & 4  &        & 5 &                  \\ 
			\vdots &	\vdots & \vdots  & \vdots  & \vdots\\ 
			n & & n & &n
			
		\end{tabular}
		
	\end{center}
	\textbf{Comment}: Suppose r$_0$ replies to p$_1$ and p$_2$ (r$_1$ hasn't replied to p$_0$ yet) respectively with 1 and 2, now pp(r$_0$) = 3
\end{frame}

\begin{frame}{Dining Philosophers}
	\begin{center}
		
		\begin{tikzpicture}[scale=.5,every node/.style={minimum size=1cm},on grid]
		
		\node (p0)[process]  at (0,2) {$p_0$};
		\node (p1)[process]  at (4,2) {$p_1$};
		\node (p2)[process]  at (8,2) {$p_2$};
		\node (r0)[resource] at (0,-1) {$r_{0}$};		
		\node (r1)[resource] at (5,-1) {$r_{1}$};
		\node (r2)[resource] at (10,-1) {$r_{2}$};
		
		\draw[requested] (r0) -- (p0);
		\draw[requested] (r1) -- (p0);
		
		\draw[requested] (r2) -- (p1);
		\draw[requested] (r0) -- (p1);
		
		\draw[requested] (r2) -- (p2);
		\draw[requested] (r0) -- (p2);
		
		
		\end{tikzpicture}
	\end{center}	
	
	\begin{center}
		
		
		\begin{tabular}{ c c| c c |c c} 
			
			&	r$_0$ &  & r$_1$ & & r$_2$   \\  
			\hline 
			0 &	 q$_0$*     & 1  &   q$_0$*      & 2 &	\\ 
			1 &	q$_1$*	  	   & 2  & 	    & 3	&			 \\ 
			2 &	 q$_2$*     	   & 3  &        & 4 &                  \\ 
			3 &	     	   & 4  &        & 5 &                  \\ 
		
			
		\end{tabular}
		
	\end{center}
	\textbf{Comment}: Suppose r$_1$ now replies to p$_0$ with 1 and increments pp(r$_1$) and now  p$_0$ could lock max(0, 1) = 1, but  r$_0$(1) is promised for someone else. Sure, it's possible that these will be released as other values might be higher, but it's likely they won't. (I'm considering locking lane with keys(q$_n$), but that's not a particularly nice solution)
\end{frame}


\end{document}
