\documentclass[10pt]{beamer}
\usepackage{tikz}
\usetikzlibrary{arrows.meta}
\usetikzlibrary{fit}

\usetikzlibrary{arrows,positioning, calc,lindenmayersystems,decorations.pathmorphing,intersections}
\tikzstyle{resource}= [draw,minimum size=16pt,inner sep=0pt]
\tikzstyle{process} = [draw,minimum size=16pt,inner sep=0pt,circle]
\tikzstyle{allocated} = [->,thick,arrows={-latex}]
\tikzstyle{requested} = [<-,thick,arrows={latex-}, dashed]

\usepackage{tikz}



\usetikzlibrary{positioning}
\newcommand{\MonetaryLevel}{Monetary level}
\newcommand{\RealLevel}{Real level}
\newcommand{\Firms}{Firms}
\newcommand{\Households}{Households}
\newcommand{\Banks}{Banks}
\newcommand{\Commodities}{Commodities}
\newcommand{\LaborPower}{Labor power}
\newcommand{\Wages}{Wages}
\newcommand{\Consumption}{Consumption}
\newcommand{\Credits}{Credits}
\newcommand{\Withdrawals}{Withdrawals}
\newcommand{\Deposits}{Deposits}
\newcommand{\Repayments}{Repayments}

\newcommand{\yslant}{0.5}
\newcommand{\xslant}{-0.6}



\newcommand{\VALID}		{\mathsf{valid}}
\newcommand{\INVALID}	{\mathsf{invalid}}
\newcommand{\UNKNOWN}	{\mathsf{unknown}}
\newcommand{\FAILED}	{\mathsf{failed}}
\newcommand{\prover}	{\mathrm{prover}}
\newcommand{\INTER}[1] 	{\llbracket #1 \rrbracket}

\providecommand{\keywords}[1]{\textbf{{Keyword:}} #1}

\usepackage{bsymb}
\input{railheader}

\newtheorem*{remark}{PhD Objective.}



\usetheme[progressbar=frametitle]{metropolis}
\usepackage{appendixnumberbeamer}

\usepackage{booktabs}
\usepackage[scale=2]{ccicons}

\usepackage{pgfplots}
\usepgfplotslibrary{dateplot}

\newcommand\blfootnote[1]{%
	\begingroup
	\renewcommand\thefootnote{}\footnote{#1}%
	\addtocounter{footnote}{-1}%
	\endgroup
}

\usepackage{xspace}
\newcommand{\themename}{\textbf{\textsc{metropolis}}\xspace}


\definecolor{darkblue}{rgb}{0,0,0.3}
\definecolor{lightblue}{rgb}{0.3,0.3,1}
\definecolor{darkred}{rgb}{0.7,0,0}
\definecolor{lightred}{rgb}{1,0.3,0.3}
\definecolor{orange}{rgb}{0.6,0.1,0.1}
\newcommand{\VAR}[1] {\textcolor{darkblue}{\mathit{#1}}}

\newcommand{\PAR}[1] {\textcolor{lightblue}{\mathit{#1}}}
\newcommand{\CST}[1] {\textcolor{lightred}{\mathsf{#1}}}
\input{convheader}

\usepackage{listings}
\usepackage{color}

\definecolor{codegreen}{rgb}{0,0.6,0}
\definecolor{codegray}{rgb}{0.5,0.5,0.5}
\definecolor{codepurple}{rgb}{0.58,0,0.82}
\definecolor{backcolour}{rgb}{0.95,0.95,0.92}

\lstdefinestyle{mystyle}{
	backgroundcolor=\color{backcolour},   
	commentstyle=\color{codegreen},
	keywordstyle=\color{magenta},
	numberstyle=\tiny\color{codegray},
	stringstyle=\color{codepurple},
	basicstyle=\scriptsize,
	breakatwhitespace=false,         
	breaklines=true,                 
	captionpos=b,                    
	keepspaces=true,                 
	numbers=left,                    
	numbersep=5pt,                  
	showspaces=false,                
	showstringspaces=false,
	showtabs=false,                  
	tabsize=2
}

\lstset{style=mystyle}

\title{Theories, Techniques and Tools for Engineering
	Heterogeneous Railway Networks}


%\date{}
\author{Paulius Stankaitis and Alexei Iliasov}
\institute{Centre for Software Reliability, Newcastle University, UK \vspace{0.2cm}\\ \begin{center} \small
		RSSRail Conference '17\vspace{0.1cm}\\ November 16th, Pistoia
	\end{center}}
\titlegraphic{\hfill\includegraphics[height=0.8cm]{Newcastle_University_Logo.pdf}}
\date{}
\begin{document}



\begin{frame}{Distributed Resource Reservation - Problems}
	\begin{center}
		
		\begin{tikzpicture}[scale=.5,every node/.style={minimum size=1cm},on grid]
		
		\node (p0)[process]  at (2,2) {$a_0$};
		\node (p1)[process]  at (6,2) {$a_1$};
		
		\node (r0)[resource] at (2,-1) {$r_{0}$};		
		\node (r1)[resource] at (6,-1) {$r_{1}$};
		
		
		
		\draw[requested] (r0) -- (p0) node [pos=0.3,left,font=\footnotesize] {q$_0$};
		\draw[requested] (r1) -- (p0) node [pos=0.5,left,font=\footnotesize] {q$_0$};
		
		\draw[requested] (r1) -- (p1) node [pos=0.3,right,font=\footnotesize] {q$_1$};
		\draw[requested] (r0) -- (p1) node [pos=0.5,right,font=\footnotesize] {q$_1$};
		
		\end{tikzpicture}
	\end{center}
	\begin{center}
		
		
		\begin{tabular}{ c | c} 
			
			&	r$_0$  \\  
			\hline 
			0 &	     	\\ 
			1 &	   \\ 
			2 &	     \\ 
			\vdots &\\ 
			n & 
			
		\end{tabular}
		\quad
		\begin{tabular}{ c | c} 
			
			&	r$_1$  \\  
			\hline 
			1 &	     	\\ 
			2 &	 \\ 
			3 &	     \\ 
			\vdots &\\ 
			n & 
			
		\end{tabular}	
		
	\end{center}
	\textbf{Remark}: Suppose, we have a scenario where a$_0$ and a$_1$ completed lane negotiation step, and now can write (create lanes) a$_0 \rightarrow 1$, a$_0 \rightarrow 2$.	
\end{frame}

\begin{frame}{Distributed Resource Reservation - Problems}
	\begin{center}
		
		\begin{tikzpicture}[scale=.5,every node/.style={minimum size=1cm},on grid]
		
		\node (p0)[process]  at (2,2) {$a_0$};
		\node (p1)[process]  at (6,2) {$a_1$};
		
		\node (r0)[resource] at (2,-1) {$r_{0}$};		
		\node (r1)[resource] at (6,-1) {$r_{1}$};
		
		
		
		\draw[requested] (r0) -- (p0) node [pos=0.3,left,font=\footnotesize] {q$_0$};
		\draw[requested] (r1) -- (p0) node [pos=0.5,left,font=\footnotesize] {q$_0$};
		
		\draw[requested] (r1) -- (p1) node [pos=0.3,right,font=\footnotesize] {q$_1$};
		\draw[requested] (r0) -- (p1) node [pos=0.5,right,font=\footnotesize] {q$_1$};
		
		\end{tikzpicture}
	\end{center}
	\begin{center}
		
		
		\begin{tabular}{ c | c} 
			
			& r$_0$	  \\  
			\hline 
			0 &	     	\\ 
			1 &	   \\ 
			2 &	 a$_0$    \\ 
			\vdots &\\ 
			n & 
			
		\end{tabular}
		\quad
		\begin{tabular}{ c | c} 
			
			&	r$_1$  \\  
			\hline 
			1 &	     	\\ 
			2 &	 \\ 
			3 &	     \\ 
			\vdots &\\ 
			n & 
			
		\end{tabular}	
		
	\end{center}
	\textbf{Remark}: When r$_0$ receives write(a$_0$, 2) messages, it stores a$_0$ at position slot 2 and since there is none else in r$_0$ request pool it will send a ready message to a$_0$ (a$_0$ is still waiting ready message from r$_1$ to starting consuming resources). 	
\end{frame}

\begin{frame}{Distributed Resource Reservation - Problems}
	\begin{center}
		
		\begin{tikzpicture}[scale=.5,every node/.style={minimum size=1cm},on grid]
		
		\node (p0)[process]  at (2,2) {$a_0$};
		\node (p1)[process]  at (6,2) {$a_1$};
		
		\node (r0)[resource] at (2,-1) {$r_{0}$};		
		\node (r1)[resource] at (6,-1) {$r_{1}$};
		
		
		
		\draw[requested] (r0) -- (p0) node [pos=0.3,left,font=\footnotesize] {q$_0$};
		\draw[requested] (r1) -- (p0) node [pos=0.5,left,font=\footnotesize] {q$_0$};
		
		\draw[requested] (r1) -- (p1) node [pos=0.3,right,font=\footnotesize] {q$_1$};
		\draw[requested] (r0) -- (p1) node [pos=0.5,right,font=\footnotesize] {q$_1$};
		
		\end{tikzpicture}
	\end{center}
	\begin{center}
		
		
		\begin{tabular}{ c | c} 
			
			& r$_0$	  \\  
			\hline 
			0 &	     	\\ 
			1 &	   \\ 
			2 &	 a$_0$    \\ 
			\vdots &\\ 
			n & 
			
		\end{tabular}
		\quad
		\begin{tabular}{ c | c} 
			
			&	r$_1$  \\  
			\hline 
			1 &	a$_1$     	\\ 
			2 &	 \\ 
			3 &	     \\ 
			\vdots &\\ 
			n & 
			
		\end{tabular}	
		
	\end{center}
	\textbf{Remark}: Now let's say the next a$_1$ write message arrives at r$_1$ and same happens again. Ready message is sent to a$_1$ by r$_1$ (a$_1$ still waits another ready message from r$_0$ to consume).
\end{frame}

\begin{frame}{Distributed Resource Reservation - Problems}
	\begin{center}
		
		\begin{tikzpicture}[scale=.5,every node/.style={minimum size=1cm},on grid]
		
		\node (p0)[process]  at (2,2) {$a_0$};
		\node (p1)[process]  at (6,2) {$a_1$};
		
		\node (r0)[resource] at (2,-1) {$r_{0}$};		
		\node (r1)[resource] at (6,-1) {$r_{1}$};
		
		
		
		\draw[requested] (r0) -- (p0) node [pos=0.3,left,font=\footnotesize] {q$_0$};
		\draw[requested] (r1) -- (p0) node [pos=0.5,left,font=\footnotesize] {q$_0$};
		
		\draw[requested] (r1) -- (p1) node [pos=0.3,right,font=\footnotesize] {q$_1$};
		\draw[requested] (r0) -- (p1) node [pos=0.5,right,font=\footnotesize] {q$_1$};
		
		\end{tikzpicture}
	\end{center}
	\begin{center}
		
		
		\begin{tabular}{ c | c} 
			
			& r$_0$	  \\  
			\hline 
			0 &	     	\\ 
			1 &	 a$_1$   \\ 
			2 &	 a$_0$    \\ 
			\vdots &\\ 
			n & 
			
		\end{tabular}
		\quad
		\begin{tabular}{ c | c} 
			
			&	r$_1$  \\  
			\hline 
			1 &	a$_1$     	\\ 
			2 &	 \\ 
			3 &	     \\ 
			\vdots &\\ 
			n & 
			
		\end{tabular}	
		
	\end{center}
	\textbf{Remark}: Now let's say a$_1$ is written to r$_0$ but ready message already has been sent to a$_0$ - so system deadlocks. 
\end{frame}

\begin{frame}{Distributed Resource Reservation - Problems}
	\begin{itemize}
		\item In the previous version of the protocol once a ready message is sent another one couldn't be sent until release message would be received. 
		\item I think allowing ready messages to be sent if someone else writes to the smaller index to the current one, still might result in unsafe situation too.
		\item Solution is to introduce few more message exchanges with pre-ready step and also prohibiting ready message sending while someone has locked a resource.
	\end{itemize}
	
\end{frame}

\begin{frame}{Distributed Resource Reservation - 2$^{nd}$ Protocol Part}

\small

	\begin{itemize}
		\item Resource keeps sending pre-ready message to agents, if:
			\begin{itemize}
				\item Read-pointer is not locked;
				\item A received write message contains the new minimum value of the request pool.
			\end{itemize}
			\item Once agent received all pre-ready message:
				\begin{itemize}
					\item it sends lock messsages to all interested resources;
				\end{itemize}
			\item If resource receives a lock message it can reply in two ways:
			\begin{itemize}
				\item READY message if read pointer still points at that agent (also stops sending pready messages).
				\item DENY message if someone wrote to the resource with the new minimum value.
			\end{itemize}
			\item Once agent receives all DENY/READY messages it can act in two ways:
			\begin{itemize}
				\item if all messages were READY then consume resources;
				\item if exists a DENY message then send unlock messages to resources which sent READY message.
			\end{itemize}
			\item If resource receives a unlock message it continues sending pre-ready messages.
	\end{itemize}
		

\end{frame}


\end{document}
