\documentclass{llncs}
\usepackage[latin1]{inputenc}
\usepackage[cmex10]{amsmath}
\usepackage{amsfonts}
\usepackage{amssymb}
%\usepackage{stmaryrd}
\usepackage{graphicx}
\usepackage{cite}
\usepackage[table]{xcolor}
\usepackage{enumerate}
\usepackage{pdflscape}
\usepackage{float}
\usepackage{tcolorbox}
\usepackage {bsymb,b2latex}
\usepackage{fancyhdr,lastpage,color}
\usepackage{pgfgantt}
\usepackage{rotating}
\usepackage[graphicx]{realboxes}
\usepackage{multicol}
\setlength{\columnsep}{1cm}
\usepackage{caption}
\captionsetup[table]{position=bottom}
\usepackage{siunitx}
\newcommand\Tstrut{\rule{0pt}{2.6ex}}         % = `top' strut
\newcommand\Bstrut{\rule[-0.9ex]{0pt}{0pt}}   % = `bottom' strut

\usepackage{tikz}

\usetikzlibrary{arrows,positioning, calc,lindenmayersystems,decorations.pathmorphing,intersections}
\tikzstyle{resource}= [draw,minimum size=16pt,inner sep=0pt]
\tikzstyle{process} = [draw,minimum size=16pt,inner sep=0pt,circle]
\tikzstyle{allocated} = [->,thick,arrows={-latex}]
\tikzstyle{requested} = [<-,thick,arrows={latex-}, dashed]


\usetikzlibrary{positioning}
\newcommand{\MonetaryLevel}{Monetary level}
\newcommand{\RealLevel}{Real level}
\newcommand{\Firms}{Firms}
\newcommand{\Households}{Households}
\newcommand{\Banks}{Banks}
\newcommand{\Commodities}{Commodities}
\newcommand{\LaborPower}{Labor power}
\newcommand{\Wages}{Wages}
\newcommand{\Consumption}{Consumption}
\newcommand{\Credits}{Credits}
\newcommand{\Withdrawals}{Withdrawals}
\newcommand{\Deposits}{Deposits}
\newcommand{\Repayments}{Repayments}

\newcommand{\yslant}{0.5}
\newcommand{\xslant}{-0.6}

\floatstyle{plaintop}
\restylefloat{table}



\newcommand{\VALID}		{\mathsf{valid}}
\newcommand{\INVALID}	{\mathsf{invalid}}
\newcommand{\UNKNOWN}	{\mathsf{unknown}}
\newcommand{\FAILED}	{\mathsf{failed}}
\newcommand{\prover}	{\mathrm{prover}}
\newcommand{\INTER}[1] 	{\llbracket #1 \rrbracket}

\providecommand{\keywords}[1]{\textbf{{Keyword:}} #1}

\usepackage{bsymb}
\input{railheader}
\input{convheader}

\newganttchartelement*{mymilestone}{
	mymilestone/.style={
		shape=isosceles triangle,
		inner sep=0pt,
		draw=cyan,
		top color=white,
		bottom color=cyan!50
	},
	mymilestone incomplete/.style={
		/pgfgantt/mymilestone,
		draw=yellow,
		bottom color=yellow!50
	},
	mymilestone label font=\slshape,
	mymilestone left shift=0pt,
	mymilestone right shift=0pt
}



\begin{document}

\title{Formal development of a cross boundary route locking and releasing protocol}


\author{Paulius Stankaitis\inst{1}, Alexei Iliasov\inst{1},  Alexander Romanovsky\inst{1} and Yamine A\"{i}t-Ameur\inst{2}}


\institute{
	Newcastle University, Newcastle upon Tyne, United Kingdom \\
	\email{\{p.stankaitis|alexei.iliasov|alexander.romanovsky\}@ncl.ac.uk} \vspace{.1cm}\\
	\and 
	INPT-ENSEEIHT, 2 Rue Charles Camichel, Toulouse, France\\
	\email{yamine@enseeiht.fr}}
	




  






\date{}

\maketitle



\begin{abstract}
The railway interlocking system is essential for ensuring safe operation of railway networks.  Nowadays a computer based system is used to establish safe routes trains can pass through. In many cases a train is required to transition from one interlocking area to another and cross boundary route locking and releasing protocol must ensure a safe transition. In this paper we present a formal development of the cross boundary route locking and releasing protocol in Event-B. The developed distributed protocol not only ensures safety properties of the interlocking transition but also guarantees the deadlock-freedom of the railway system. %Furthermore we consider a general version of the protocol.

\end{abstract}

\begin{keywords}
	Distributed Railway Interlocking $\cdot$ Event-B $\cdot$ Formal Verification $\cdot$ Stepwise Development $\cdot$ Safety Verification $\cdot$ System Liveness

\end{keywords}

\section{Introduction}
The railway interlocking system is essential for ensuring safe operation of railway networks. Nowadays a computer based system is used to establish safe routes trains can pass through. A railway intelocking system is capable of controlling only a portion of railway network therefore multiple interlocking systems have to be used to guarantee safety of the whole network. Often a train is required to transition from one interlocking area to another and cross boundary route locking and releasing protocol must ensure a safe transition. Because of distributed system complexity rigourous and systematic verification methods have to be used to guarantee correctness of such protocols.

Formal methods - a mathematical model driven methods provide a systematic approach for developing complex systems. They offer an approach to specify systems precisely via a mathematically defined syntax and semantics as well as formally validate them by using semi-automatic or automatic techniques. In spite of formal methods success in the railway domain considerably little has been done in considering distributed nature of the railway interlocking. In this paper we present a formal development of the cross boundary route locking and releasing protocol in the Event-B \cite{EventBBook} specification language. 

The main protocol objective is to guarantee a safe rolling stock transition from one interlocking area to the following one. In addition to safety properties one should also guarantee liveness and deadlock-freedom of the system as well. The literature review (see below) showed that based on the engineering concept two main distributed railway models exist. The first concept of the interlocking system (described in the first paragraph) controls a number of railway agents and is more closer to real-world interlocking systems. Whereas the second is more fined grained interlocking system where each component has an autonomy to make a decision. %\textbf{Our approach is closer to the first one..}


In the following section we describe Event-B specification language which was used to develop formal model and prove the properties of interest. In Section \ref{section3} we informally describe the cross boundary route locking and releasing protocol including key requirements and environment assumptions. In Sections \ref{fmodel} - \ref{proof} we present the formal model of the protocol and verification challenges. In the last section we discuss the modelling and outline concluding remarks. \\


\noindent \textbf{Related Work.} To authors knowledge the earliest attempt to formally analyse distributed railway solid-state interlocking systems was completed by Morley \cite{morley1996safety}. In this interesting work author developed a formal model of a protocol for a cross boundary route locking and releasing mechanism. By analysing temporal properties of the model he discovered that in certain scenarios safety properties can be violated. Few years later a paper by Cimatti et al. \cite{cimatti1999formal} presented an industrially driven formal methods study where authors formally modelled a communication protocol for safety-critical distributed systems including distributed railway interlocking systems. Their method used Statecharts diagrams to specify high level protocol properties and the \scriptsize{OBJECT}\normalsize{GEODE} model checker for the protocol validation. 

In other work a different concept of distributed railway control system was introduced by Haxthausen and Peleska \cite{haxthausen2000formal}. Their presented engineering concept of the control system relied on a radio based communication and switch boxes - systems which can only control a single railway switch. Authors formally modelled the system with the RAISE \cite{george1995raise} specification language  which allowed to develop a formal model incrementally using a refinement process and prove refinement and safety properties with available justification tools. The timing properties of the design were considered in the extended work \cite{madsen2005modelling}. Similar ideas for distributed railway interlocking system were also presented in \cite{banci2004role, hei2008toward} where authors used Statecharts and Petri Nets to model and verify decentralised railway interlocking. 
 \newpage

\section{Distributed Resource Allocation Protocol Description}
\label{section3}

The objective of the protocol is to enable distributed atomic reservation of a collection of resources, for instance, two distinct agents $A_0$ and $A_1$ may require any resource collections $r_1$ and $r_2$ where $r_1, r_2 \subseteq R$. The protocol must guarantee that each agent gets all or nothing - partial request satisfaction is not permitted and ensure that every agent request will be eventually satisfied as long as certain degenerate situations are avoided. 

A resource itself has an attributed memory where requests can be stored also a read pointer $\mathsf{rp}(r_k)$ and a promise pointer $\mathsf{pp}(r_k)$ - the largest {promised} index in the $r_k$ request pool. The initial value $\mathsf{pp}(r_k)$ is unique for every request pool and is assigned statically. In our protocol concept a resource is only allowed to exchange messages with agents (and agents only with resources). Since we do not consider degenerate or mallicious situations messages cannot be altered or lost but requests can arrive at resources in any order.

\begin{figure}
		\begin{center}
			\begin{tabular}{ c c | c c | c c | c c } 
			\hspace{.1cm}  r$_0$ \hspace{.1cm} & &  \hspace{.1cm} r$_1$ \hspace{.1cm} & & \hspace{.1cm} r$_2$ \hspace{.1cm}  & & \hspace{.1cm} r$_3$ \hspace{.1cm} \\[1mm]  
			\hline  	
		  	\Tstrut \hspace{.1cm} 0 \hspace{.1cm} & \textcolor{white}{a$_0$} \hspace{.1cm} &  \hspace{.1cm} 0 \hspace{.1cm} & ${A_0}^*$ \hspace{.1cm} &  \hspace{.1cm} 0 \hspace{.1cm}  & ${A_1}^*$ \hspace{.1cm} & \hspace{.1cm} 0 \hspace{.1cm} & \textcolor{white}{$a_0$} \hspace{.1cm} \\[1mm]
		  	 \hspace{.1cm} 1 \hspace{.1cm} & &  \hspace{.1cm} 1 \hspace{.1cm} & ${A_1}^*$ \hspace{.1cm} &  \hspace{.1cm} 1 \hspace{.1cm}  & ${A_0}^*$ \hspace{.1cm} & \hspace{.1cm} 1 \hspace{.1cm} \\[1mm]
		  	  \hspace{.01cm} \vdots \hspace{.1cm} & &  \hspace{.01cm} \vdots  \hspace{.1cm} & &  \hspace{.01cm} \vdots  \hspace{.1cm}  &  & \hspace{.01cm} \vdots  \hspace{.1cm} \\[1mm]
		  	 \hspace{.1cm} n \hspace{.1cm} & &  \hspace{.1cm} n \hspace{.1cm} & &  \hspace{.1cm} n \hspace{.1cm}  &  & \hspace{.1cm} n \hspace{.1cm} 
		\end{tabular}
		
	\end{center}
		\caption{Distributed resource allocation blocking scenario. An asterisk symbol indicates that a slot has been promised but not locked for the agent.}
	\label{lane1}
\end{figure}

Permitting request arrival delays can cause situations where different requests are blocking each other and causes the system to livelock. A blocking scenario is depicted in Figure \ref{lane} where agents ${A_0}$ and ${A_1}$ have both requested resources ${(r_1, r_2)}$ and they were promised slots ${(1, 2)}$ and ${(2, 1)}$ in ${(r_1, r_2)}$ respectively. Consequently if both agents would go ahead and lock these slots agents ${A_0}$ and ${A_1}$ would block each other and their requests could never be satisfied as partial request satisfaction is not permitted. 

\begin{figure}
	
	
	\begin{center}
		
		
		\begin{tabular}{ c | c | c | c | c | c } 
			
			& \hspace{.1cm} r$_0$ \hspace{.1cm} &   \hspace{.1cm} r$_1$ \hspace{.1cm} &  \hspace{.1cm} r$_2$ \hspace{.1cm}  &  \hspace{.1cm} r$_3$ \hspace{.1cm} \\[3mm]  
			\hline  
			\hspace{.05cm}&	\Tstrut	\hspace{.1cm} 		0 &     \hspace{.1cm} 1  &        \hspace{.1cm} 2 &  \cellcolor{gray!65}\hspace{.1cm} 3  \\[1mm]
			
			\hspace{.05cm}&		\hspace{.1cm}	    	1 &		  	   \hspace{.1cm} 2  & 	     \cellcolor{gray!65}\hspace{.1cm} 3 &  \cellcolor{gray!45}\hspace{.1cm} 4 \\[1mm]
			
			\hspace{.05cm}&		\hspace{.1cm}	    	2 &	      	   \cellcolor{gray!65}\hspace{.1cm} 3  &   \cellcolor{gray!45}\hspace{.1cm} 4 &  \cellcolor{gray!25}\hspace{.1cm} 5  \\[1mm]
			
			\cellcolor{gray!65}	$l_0$\hspace{.05cm}&\cellcolor{gray!65}	\hspace{.1cm} 3 & \cellcolor{gray!45}\hspace{.1cm} 4  &  \cellcolor{gray!25}\hspace{.1cm} 5 &\cellcolor{gray!10}\hspace{.1cm} 6  \\[1mm]
			
			\cellcolor{gray!45}	$l_1$\hspace{.05cm}&\cellcolor{gray!45}		\hspace{.1cm}			4 &	      	   \cellcolor{gray!25}\hspace{.1cm} 5    &\cellcolor{gray!10}\hspace{.1cm} 6    \\[1mm] 
			
			\cellcolor{gray!25}	$l_2$\hspace{.05cm}&\cellcolor{gray!25}		\hspace{.1cm}		    5 &	     \cellcolor{gray!10}\hspace{.1cm} 6                  \\[1mm] 
			
			\cellcolor{gray!10}	$l_3$\hspace{.05cm}&\cellcolor{gray!10}	\hspace{.1cm} 	6   	                  \\[1mm]
			
		\end{tabular}
		
	\end{center}
	
	\caption{An example virtual distributed lane data structure}
	\label{lane}
\end{figure}


The principal mechanism of a solution we offer to prevent blocking situations and ensure progress is \textit{distributed lane} - a virtual data structure which is only present at a conceptual level. To be more specific a distributed lane is a distributed data structure made of at first approximation a collection of resource request pools one per each resource where columns represent resources request pools and rows represent lane. 

To lock (form a lane) a resource an agent has to go through a number of steps defined by the protocol described below. \\
%Request pools are modelled explicitly with one private pool per resource. 


\noindent \textbf{Request.} An agent $A_i$ which intends to lock a set of resources $res \subseteq R$ generates a request to request pools associated with resources $\mathbf{r}$. Such requests are sent and received in no particular order and contain only agent name $A_i$. We define such message as $\mathsf{request(A_i)}$ and write $\mathsf{request(A_i) \rightarrow r_k}$ to state it is addressed to resource $r_k \in \mathbf{r}$. \\


\noindent \textbf{Reply.} Once a request pool $r_k$ receives request $\mathsf{request(A_i)}$ it replies with a message $\mathsf{reply(\mathsf r_k, {pp}(r_k))} \rightarrow A_i$ and then increments $\mathsf{pp}(r_k)$. \\

\noindent \textbf{ConfirmWR.} After sending all $\mathsf{request(A_i)}$ messages an agent $A_i$ awaits for all replies to arrive which carry values $\mathsf{pp}(r_k)$. Depending on these values following actions should be taken: \\

\textbf{Write.} If all reply values on reception are equal then $A_i$ should write at index $n$ to request pools $\mathsf{write(A_i, n) \rightarrow r_k}$. \\

\textbf{sRequest.} If all values on reception are not equal then the agent must renegotiate a new index. This time an agent sends new (special) requests to a subset of resources. We define such message as \textbf{$\mathsf{srequest(A_i, max) \rightarrow r_k}$} where $r_k$ is $\mathsf{r_k \subset \mathbf{r}}$ and must satisfy $\mathsf{\forall r \cdot r \in r_k \Rightarrow reply(r_k)} < \mathsf{max(replies(A_i))}$.  \\

\textbf{sReply.}  Once a request pool $r_k$ receives a special request it replies with the following message $\mathsf{reply(\mathsf r_k, max)} \rightarrow A_i$ where $\mathsf{max}$ the maximum value of $\mathsf{pp(r_k)}$ and received $\mathsf{srequest(A_i)}$. \\

\noindent \textbf{pReady.} A pre-ready message is sent by a resource to inform an agent that it is available for consumption and we define such message $\mathsf{pready(\mathsf r_k)} \rightarrow A_i$.\\

\textbf{pReady (wr).} \\

\textbf{pReady (rl).} \\

\noindent \textbf{Lock.} An agent waits for all pre-ready messages to arrive and once it receives them it sends a lock message to resources as follows $\mathsf{lock(\mathsf A_i)} \rightarrow r_k$.\\

\noindent \textbf{Respond.} A request pool $r_k$ which receives a lock message will respond with a message $\mathsf{respond(\mathsf r_k, response)} \rightarrow A_i$ where $\mathsf{response \in \{confirm, deny\}}$. \\

\noindent \textbf{Decide.} An agent waits for all respond messages to arrive and depending on these messages following actions will be taken. \\

\textbf{Unlock.} If one of the messages is a deny message, an agent $\mathsf{A_i}$ will send an unlock messages to all resources which replied with confirm message. \\

\textbf{Consumption.} If all messages were confirm messages, an agent $\mathsf{A_i}$ can proceed with resource consumption.\\

\noindent \textbf{Release.} An agent $A_i$ will eventually release a resource by sending a message to to resource. \vspace{1cm} \\


%\noindent \textbf{Ready.} When some $A_y$ releases a resource $r_k$, read pointer is updated to the next minimum value of the request pool $\mathsf{rp}(r_k)$ and a ready message is sent to subsequent agent in the request pool $\mathsf{ready(\mathsf{r_k})} \rightarrow A_i$.  \\

%Once all messages are received, the agent computes $n=max_{r \in \mathbf{r}} \mathsf{pp}(r)$. This value is used to generate message $\mathsf{reserve(A_i, n)} \rightarrow r, r \in \mathbf{r}$, carrying maximum value $n$ and addressed to all the pools.

%\subsection*{Step 3-4. Accept/Reply (Paulius)}

% In this version one can combine Steps 3 and 4. Since the approach of this protocol version is to iteratively try to form a lane. A basic solution could be as follows:

%\begin{enumerate}
%	\item After receiving all replies one can check  $\mathsf \forall e \cdot e \in reply[\{A_i\}]^{-1} \Rightarrow e = n$
%	\begin{enumerate}
%		\item If true proceed to Step 5a.
%		\item Else prooced to Step 5b.
%	\end{enumerate}
%\end{enumerate}

%\textbf{Remark}: In order to allow $\mathsf \forall e \cdot e \in reply[\{A_i\}]^{-1} \Rightarrow e \leq n$ one must make an assumption on request arriving frequency. 

%\noindent \textbf{Write (request).} Agent $A_i$ again awaits for all replies to arrive. In a general case, these are made of  $\mathsf{accept}(n)$ (all $n$ values are same by the nature of the protocol) and  $\mathsf{reject}$ messages. This step applies only to the situation when there are no rejections.

%Since all the pools replied with the acceptance, it is now possible to write into all the pools and thus form a new lane. The agent replies with message $\mathsf{write}(A_i, n) \rightarrow r_k$ to each pool $r_k$.


%The protocol steps above describe how to form lanes from distributed pools. The pragmatics of such lanes is atomic reservation of resource sets. The lanes semantics guarantees, as we shall prove in the model, the following two properties:

%\begin{enumerate}
%	\item every reserved value $A_i$ forms a valid lane: it has same index in each request pool;
%	\item the smallest index of all request pools (i.e., the lane with a smallest index) define a valid set of requested resources that can be locked;
%	\item there are no deadlocks.
%\end{enumerate}


%\noindent \textbf{System Requirement 1.} Cross boundary route locking and releasing protocol must ensure that a cross boundary route has been reserved only to a single train at a time. \\

%\noindent \textbf{System Requirement 2.} Cross boundary route locking mechanism must ensure that a locked cross boundary route has points properly positioned and signals sets.



In addition to safety and deadlock-freedom one must ensure agent starvation freedom property in developing a distributed protocol. A protocol must ensure that agents will eventually be allocated resources which have been requested. A starvation could potentially occur with our protocol if multiple agents require similar resources and because of diagonal blocking agents would continuously need to renegotiate a new lane.



%The protocol steps above describe how to form lanes from distributed pools. The pragmatics of such lanes is atomic reservation of resource sets. The lanes semantics guarantees, as we shall prove in the model, the following two properties:

%\begin{enumerate}
%	\item every reserved value $A_i$ forms a valid lane: it has same index in each request pool;
%	\item the smallest index of all request pools (i.e., the lane with a smallest index) define a valid set of requested resources that can be locked;
%	\item there are no deadlocks.
%\end{enumerate}

\subsection{Refinement Plan} \newpage

%\section{Event-B}

%The Event-B formal specification language has been used a number of times in developing railway models . 
The Event-B mathematical language used in the system development and analysis is an evolution of the classical B
method \cite{abrial2005b} and Action Systems \cite{ActionSystems}. %Perhaps due to the success of the B method and a good tool support Event-B has also been a popular language choice for modelling railway systems \cite{butler2002system, EventBBook, kiss2016developing}. 
The formal specification language offers a fairly high-level mathematical language
based on a first-order logic and Zermelo-Fraenkel set theory as well as an economical yet expressive modelling notation.
The formalism belongs to a family of state-based modelling languages where a state of a discrete system is simply a
collection of variables and constants whereas the transition is a guarded variable transformation. 

\begin{figure}[h]
	\footnotesize{
		\centering
		\EventBSystem{M}{
			\EventBSees{Context}
			\EventBVarsC{v}
			\EventBInvC{I(c, s, v)}
			\EventBInitC{R(c, s, v')}
			\EventBEvents{
				\EventBAnyC{E_1}{vl}{g(c, s, vl, v)}{S(c, s, vl, v, v')}
				\dots \\
			}
		}  
	}
	\caption{Event-B machine structure.}
	\label{EventB_structure}
\end{figure}

A cornerstone of the Event-B method is the stepwise development that facilitates a gradual design of a system implementation through a number of correctness preserving refinement steps. The model development starts with a creation of a very abstract specification and the model is completed
when all requirements and specifications are covered. The Event-B model is made of two key components - machines and contexts which respectively describe dynamic and static parts of the system. The context contains modeller
declared constants and associated axioms which can be made visible in machines. The dynamic part of the model contains
variables which are constrained by invariants and initialised by an action. The state variables are then transformed by actions which are part of events and the modeller may use predicate
guards to denote when event is triggered (see Fig. 1). 

Specifying a model is not sufficient one must provide evidence about the correctness of the model as well. The Event-B
method is a proof driven specification language where model  correctness is demonstrated by generating and discharging
proof obligations - theorems in the first-order logic. The model is considered to be correct when all proof obligations
are discharged. To facilitate the formal development and verification of Event-B models the Rodin Platform \cite{RodinPlatform} was developed. In recent years a number of
useful plug-ins were developed for the Rodin Platform to automate verification and reduce modelling efforts.







%At the same time Andr{\'e} Platzer introduced an alternative approach to exploring a state-space with model checkers in verifying systems safety. A developed formalism and logic for reasoning about hybrid systems uses a deductive verification and can be implemented in a KeyMaera X verification tool\cite{platzer2008differential, platzer2008keymaera}. The later work presented  a case study where differential dynamic logic was applied for a safety verification of the European Train Control System \cite{platzer2009european}. Differential Dynamic Logic was also used to model and verify a handover protocol between two trackside train control systems (radio-block centres) by Liu et al. \cite{liu2011formal}. In a work by Cimatti et al. \cite{Cimatti2009} authors proposed a different logic based on the temporal logic with regular expressions. Their motivation was driven by a need of the automatic verification method for verifying hybrid requirements for hybrid railway system. A more recent work by Iliasov et al. \cite{iliasov2014unified} proposed a domain specific language - Unified Train Driving Policy. The formal notation allows to express both static and dynamic properties of railway in readable syntax which can be interpreted by railway engineers without prior knowledge of formal methods. A few recent formal methods projects on cyber-physical systems applied their novel techniques for modelling and verification of hybrid railway systems  \cite{intocps, advanced, atr2}. 

 \newpage


%\section{Formal Model in Event-B/5p.}
\label{fmodel}


\subsection{Distributed Lane}
\begin{footnotesize}
	\begin{description}
		
		\EVT {detect\_contention}
		\begin{description}
			\AnyPrm 
			\begin{description}
					\Item{f }
					\Item{r }
					\Item{n } 
			\end{description}
			\WhereGrd \vspace{-.3cm }
			\begin{description}
			\nItem{ grd1 }{ r \in  req }
			\nItem{ grd2 }{ f \in  req\_forks(r) }
			\nItem{ grd3 }{ cnt[\{ r\} ] = req\_forks(r) }
			\nItemY{ grd4 }{ card(req\_rep(r)) = 1 }{} 
			\nItem{ grd5 }{ n \in  req\_rep(r) }
			\end{description}
			\ThenAct \vspace{-.3cm }
			\begin{description}
				\nItem{ act1 }{ cnt :=  cnt \bunion  (req \ranres  p) }
			\end{description}
			\EndAct
		\end{description}
	\end{description}
\end{footnotesize}


\subsection{Localisation}

\subsection{Railway} \newpage

%\section{Protocol Validation (SafeCap+ProB+Proof)/1.5p}
\label{proof}

\begin{table}
	
	\centering
	\begin{tabular}{ c | c  | c | c  }
		
		Model & \hspace{.2cm} Total \hspace{.2cm} & \hspace{.2cm} Automatically \hspace{.2cm} & \hspace{.2cm} Interactively \hspace{.2cm}\\
		\hline
		Initial Model & 276   & 24   & 4 \\
		1st Refinement & 276  & 24   & 4 \\
		2nd Refinement & 276  & 24   & 4 \\
		3rd Refinement & 276  & 24   & 4 \\
		\hline
		Total & & &
	\end{tabular}
	\label{statistics}
	\caption{Tool statistics} 
\end{table} \newpage

%\section{Discussion and Conclusions/1.5p}
\label{disscusion} \newpage




















\noindent \textbf{Acknowledgments.} This work is supported by an iCASE studentship (EPSRC and Siemens Rail Automation). We are grateful to our colleagues from Siemens Rail Automation for invaluable feedback.


\bibliographystyle{plain}
\bibliography{main}





\end{document}